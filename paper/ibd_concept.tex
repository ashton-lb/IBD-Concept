\documentclass[12pt]{article}
\usepackage[english]{babel}
\usepackage[utf8]{inputenc}
\usepackage{amsmath}
\usepackage{geometry}
\geometry{a4paper, margin=2.5cm}

\title{The IBD Concept: A Mathematical and Adaptive Framework for Decision-Making Under Uncertainty}
\author{Ashton Liam Brown}
\date{2026}

\begin{document}
\maketitle

\begin{abstract}
This paper presents the IBD (Integrated Bio-Digital) concept, a theoretical framework that integrates classical mathematical optimization with adaptive machine learning techniques to support decision-making in complex productive systems under uncertainty. The approach emphasizes interpretability, probabilistic optimal regions, and robustness to biological and environmental variability.
\end{abstract}

\section{Introduction}
Decision-making in productive systems often relies on heuristics or static recommendations, despite the nonlinear and multivariate nature of the underlying processes. This work proposes a conceptual framework that combines calculus-based optimization with adaptive learning to address this gap.

\section{Mathematical Modeling}
Let the system output be defined as:
\[
y = f(x_1, x_2, \dots, x_n, t) + \varepsilon
\]
where $x_i$ are controllable variables, $t$ represents time, and $\varepsilon$ captures stochastic variability.

\section{Optimization Framework}
Optimal regions are identified by analyzing first- and second-order derivatives of the response function. Rather than computing a single optimal point, the framework defines probabilistic optimal intervals.

\section{Adaptive Layer}
Machine learning techniques are introduced to dynamically adjust model parameters based on observed data, preserving the interpretability of the mathematical structure.

\section{Conclusion}
By integrating mathematical rigor with adaptive learning, the IBD framework offers a structured approach to decision-making under uncertainty.

\end{document}
\section{Implementation Prototype}

A minimal implementation prototype was developed to validate the conceptual
structure of the IBD framework.

The prototype is implemented in Python and focuses on interpretability rather
than performance. Objective functions are evaluated over continuous domains,
with numerical derivatives used to identify stable regions of near-optimality.

Robustness is introduced by evaluating objective functions under stochastic
perturbations, allowing the framework to model uncertainty explicitly.
Additionally, the implementation follows a machine-learning-compatible
interface, including \textit{fit}, \textit{predict}, and \textit{loss} methods,
facilitating future integration with learning-based systems.

The full prototype is available in the public repository accompanying this work.
